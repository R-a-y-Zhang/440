\documentclass{article}
\begin{document}
\title{A-Star Project Summary}
\author{Rui Zhang - rz187}
\maketitle
\section{Introduction}
\subsection{Goal}
Implement and visualize the A-star pathfinding algorithm and its derivatives,
and write a program that was modular (or ``abstract''). Implement A-star with
a variety of heuristics, both admissible and inadmissible.
\subsection{Goals Achieved}
\begin{enumerate}
    \item Implemented A-star with the the Manhattan Heuristic and able to find
        the most efficient path
    \item Implemented a system for selectively choosing which A-star derivative
        to use as well as grid size, size of obstacles, and the number of obstacles
    \item Implemented a crude visualizer of the path that the program selected
    \item Implemented all parts of the assignment (to the best of my understanding)
    \item Implemented a variety of other heuristics that were either derivatives
        of the manhattan heuristic
\end{enumerate}
\section{Heuristics}
\begin{enumerate}
    \item Manhattan
    \item Uniform cost (value returned is always 0)
    \item Weighted Manhattan (manhattan distance multiplied by a constant value (2))
    \item Exponential
    \item Square root
\end{enumerate}
\section{Optimizations}
Very few optimizations were performed, if any. Beyond the use of a priority queue, for
the open set since insertion is O(log n) and ability to perform heapify in O(log n) time,
and using a set for the closed set since insertion and retrieval are both O(1), there
were no other optimizations that I was aware of. In fact, everything was par-for-the-course,
really. All algorithms were written as-is. Space complexity was also neglected, or not really
considered since it was doubtful that the program would run out of space. The only
``optimizations'' was the reduction of code complexity by modularizing a good number of parts
of the program, thus reducing code complexity. Use of classes allowed for passing smaller
groups of parameters to functions. Other than that, there is not a lot of optimizations done
that I know of. The program ran in a reasonable amount of time and that was my benchmark.
\section{Why It Was Efficient}
\begin{enumerate}
    \item A lot of these are language-specific (i.e. these optimizations are due to Python's nature)
    \item Even though a lot of areas of the program could be simplified through the use of recursion,
        vanilla Python is does not support TCO (Tail-Call Optimization) therefore, by resorting
        to loops it eliminates exceeding the call stack.
    \item Prioritizing arrays over objects reduces program overhead. Even a very simple object in
        Python has a lot of overhead. Therefore, by using objects only when absolutely necessary,
        it allowed for preserving main memory space.
    \item List comprehensions in python have a speed advantage over other methods. Therefore,
        by making use of it at various parts of the program it allowed for a faster implementation.
    \item Keeping the number of parameters passed to each function allowed for the amount of global data
        to be small by allowing a larger stack frame, which is not an issue since stack frames are
        removed after a functionis completed.
\end{enumerate}
\section{Difficulties}
\begin{enumerate}
    \item The pseudocode provided was quite confusing, but that was most likely due to me
        being unfamiliar with the set theory syntax
    \item Python, with its large and expansive library unfortunately did not have a min-heap
        implementation for objects so I had to write my own \footnote{There probably was but
        I could not find it or I failed to grasp that it was meant for that purpose}
    \item Unfortunately, I am not familiar with any visualization libraries that Python has to
        offer (like pyplot or mathplotlib) or in any other langauge so I had to resort to
        HTML/CSS (which is not my strong suit)
    \item My partner never delivered on what he said he would which certainly slowed down
        progress. In the end, I crammed as much as I could possibly do into it.
\end{enumerate}
\end{document}
